% Intended LaTeX compiler: pdflatex
\documentclass[11pt]{article}
\usepackage[utf8]{inputenc}
\usepackage[T1]{fontenc}
\usepackage{graphicx}
\usepackage{longtable}
\usepackage{wrapfig}
\usepackage{rotating}
\usepackage[normalem]{ulem}
\usepackage{amsmath}
\usepackage{amssymb}
\usepackage{capt-of}
\usepackage{hyperref}
\usepackage{listings}
\usepackage{xcolor}
\usepackage[french, american]{babel}
\author{Fabrice Niessen}
\date{2023-07-08}
\title{Vacances à la plage}
\hypersetup{
 pdfauthor={Fabrice Niessen},
 pdftitle={Vacances à la plage},
 pdfkeywords={},
 pdfsubject={},
 pdfcreator={Emacs 29.1 (Org mode 9.6.6)}, 
 pdflang={English}}
\begin{document}

\maketitle
\setcounter{tocdepth}{2}
\tableofcontents


\section{Objectif}
\label{sec:orgfb58387}

Il faut ``profiter''.

Il faut « profiter ».
<<  «
valises avec soin et partons en vacances l'esprit tranquille !

\section{Préparation des bagages}
\label{sec:org375ac94}

Pour que tout se déroule sans stress et que nous puissions profiter pleinement
N'oublions, de
nous détendre et de créer des souvenirs inoubliables. Alors, préparons nos
valises avec soin et partons en vacances l'esprit tranquille !


Pour que tout se déroule sans stress et que nous puissions profiter pleinement
N'oublions, de
nous détendre et de créer des souvenirs inoubliables. Alors, préparons nos
valises avec soin et partons en vacances l'esprit tranquille !


Pour que tout se déroule sans stress et que nous puissions profiter pleinement
de notre séjour, voici quelques conseils.

\subsection{{\color{red}\textbf{\textsc{\textsf{TODO}}}} Suggestions}
\label{sec:orgc7f7374}

\begin{itemize}
\item Trions les vêtements par type et par couleur pour faciliter le rangement.
\item Pensons aux activités prévues et apportons tout ce dont nous aurons besoin.
\item Vérifions les restrictions de bagages de la compagnie aérienne.
\end{itemize}

ezarez

\subsection{{\color{red}\textbf{\textsc{\textsf{STRT}}}} Checklist}
\label{sec:org157f2c9}

Checklist des \textbf{choses à ne pas oublier} d'emporter dans nos bagages : maillots de
bain, chapeau ou casquette, lunettes de soleil, crème solaire, brosse à dents /
dentifrice, chargeur de téléphone portable, billets de train, réservations
d'hôtel, argent liquide, parapluie (au cas où la météo jouerait des tours).

\section{Astuces pour des vacances réussies}
\label{sec:org378e8ac}

\begin{itemize}
\item Profitons de chaque instant ensemble.
\item Créons des souvenirs inoubliables.
\end{itemize}

\section{Pour les enfants}
\label{sec:org9c3aacc}

Lors de nos déplacements, il est essentiel d'emporter des activités appropriées
à l'âge de nos enfants pour les divertir pendant les trajets ou les temps
d'attente. Assurons-nous également d'avoir en quantité suffisante de la
nourriture et des boissons pour répondre à leurs besoins. Enfin, n'oublions pas
d'emporter avec nous les médicaments ou traitements médicaux nécessaires pour
nos enfants, afin de garantir leur bien-être pendant notre voyage.
\end{document}