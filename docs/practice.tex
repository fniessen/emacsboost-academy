% Intended LaTeX compiler: pdflatex
\documentclass[11pt]{article}
\usepackage[utf8]{inputenc}
\usepackage[T1]{fontenc}
\usepackage{graphicx}
\usepackage{longtable}
\usepackage{wrapfig}
\usepackage{rotating}
\usepackage[normalem]{ulem}
\usepackage{amsmath}
\usepackage{amssymb}
\usepackage{capt-of}
\usepackage{hyperref}
\usepackage[french, english]{babel}
\usepackage{xcolor}
\usepackage{listings}
\author{John Doe}
\date{\today}
\title{Fichier d'entraînement EmacsBoost}
\hypersetup{
 pdfauthor={John Doe},
 pdftitle={Fichier d'entraînement EmacsBoost},
 pdfkeywords={},
 pdfsubject={},
 pdfcreator={Emacs 30.2 (Org mode 9.7.11)}, 
 pdflang={English}}
\begin{document}

\maketitle
\tableofcontents

Ce fichier est utilisé pour s'exercer aux commandes d'édition de base et
avancées d'Emacs.  Tu peux te déplacer, modifier du texte, chercher et
bien d'autres fonctionnalités.

Le mot éditeur apparaît plusieurs fois dans ce paragraphe. Emacs est un très
puissant éditeur. Certaines personnes l'appellent un éditeur de texte, d'autres
l'appellent un éditeur programmable ou même un système d'exploitation déguisé en
éditeur.  Quand tu auras terminé ces exercices, tu auras probablement
l'impression que ton éditeur est devenu un véritable traitement de texte.

Essaie de remarquer   comment    les espaces   supplémentaires
apparaissent   dans cette phrase.  Certains mots ont   des   doubles   ou
triples espaces   avant ou après eux.
Nettoie tout cela pour que la phrase n'utilise plus qu'un espacement normal.

Voici un paragraphe avec des majuscules et minuscules mélangées à propos de ton
passe-temps favori. peut-être que tu aimes programmer, ou Jouer du piano, ou
faire de la randonnée en Montagne. emacs peut t'aider à prendre des notes,
Planifier des voyages, et écrire du code. utilise ce texte pour t'exercer
à mettre les mots en majuscules, minuscules et Capitaliser les mots avec les
commandes Emacs.
\section{Liste de courses}
\label{sec:org02d46cf}

milk
Eggs
butter
dark chocolate
pasta
tomatoes
olive oil
Coffee
butter
MILK
banana
apple
orange
kiwi
apple
mango
grape
banana
pear
fig
grapefruit
kiwi
plum
plum
date

Utilise cette liste pour t'exercer à :
\begin{itemize}
\item te déplacer par mots et par lignes
\item changer la casse des mots
\item trier les lignes
\item supprimer les lignes en double
\item enregistrer et rejouer des macros
\end{itemize}
\section{Court poème ou citation}
\label{sec:orgcfd0617}

the cursor moves, the buffer stays
keystrokes dance in quiet ways
from line to line, from word to word
silence kept, yet commands heard
\section{Données en colonnes pour édition par rectangle}
\label{sec:orge80c138}

ID   Name              Language      Score
001  Alice Johnson     Python        87
002  Bob Smith         C++           92
003  Carla Gomez       Emacs Lisp    78
004  Daniel Ito        Java          88
005  Eva Müller        Rust          95
006  Farid Khan        Python        83
007  Grace Lee         JavaScript    90
008  Henry O'Neil      Emacs Lisp    80

Indications :
\begin{itemize}
\item Ajoute un préfixe comme ``\# '' à toutes les lignes de données avec une commande
sur les rectangles.
\item Insère des numéros séquentiels au début de chaque ligne.
\item Aligne la colonne Language différemment.
\item Essaie d'ajouter un horodatage au début de chaque ligne avec une macro.
\end{itemize}
\section{Adresses email dans des formats incohérents}
\label{sec:org429b62a}

John Doe <john.doe@example.com>
jane.smith@example.org
``Boss, Big'' <big.boss@corporate.example>
lucy@example.com (Lucy)
martin dupont <martin.dupont@example.fr>
ALICE@example.com
bob.builder at example dot com
No Name <noname@example.net>
first.last@example.com
\begin{enumerate}
\item Random Hacker <jrh@example.edu>
\end{enumerate}

Utilise cette section pour t'exercer à :
\begin{itemize}
\item la recherche et le remplacement
\item des macros clavier qui normalisent les formats d'email
\item les opérations de coupe et collage (kill et yank)
\item les éditions en rectangle sur la partie domaine
\end{itemize}
\section{Paragraphe désordonné pour narrowing, remplissage et nettoyage}
\label{sec:orgc700649}




Emacs Boost


contredire

\end\{
\section{Petit ext}
\label{sec:orgfc4af6d}

\begin{lstlisting}[language=Python,numbers=none]
def add(a,  b):
 """Return the sum of a and b."""
 total = a  +  b
 return  total


def greet(name):
    message = "Hello, "  +  name + "!"
    print(message)


if __name__ == "__main__":
    result = add(2,  3)
    greet("Emacs user")
    print("Result:",   result)
\end{lstlisting}

Utilise ce code pour :
\begin{itemize}
\item te déplacer par mots, symboles et lignes
\item t'exercer à modifier l'indentation
\item essayer de commenter et décommenter des régions
\item expérimenter avec des macros qui ajustent l'espacement autour des opérateurs
\end{itemize}
\section{Petit extrait Emacs Lisp}
\label{sec:org46a77c1}

\begin{lstlisting}[language=Lisp,numbers=none]
(defun emacs-boost-hello ()
  "Simple function used for EmacsBoost practice."
  (interactive)
  (message "Hello from EmacsBoost practice file!"))

(defun emacs-boost-add (x y)
  "Return the sum of X and Y."
  (+ x y))

;; Try evaluating these functions, changing their names,
;; and adding your own variants.
\end{lstlisting}
\section{Liste de tâches façon Org (texte simple)}
\label{sec:orge1eca75}

{[} ] Installer un nouveau thème
{[} ] Activer which-key mode
{[} ] Configurer avy pour une navigation rapide
{[} ] Créer un simple yasnippet
{[} ] Mettre en place un serveur Emacs et utiliser emacsclient
{[} ] Apprendre aujourd'hui un nouveau raccourci clavier
{[} ] Revoir tous les exercices à la fin de la semaine

Utilise cette section pour :
\begin{itemize}
\item basculer manuellement les crochets
\item t'exercer à des macros qui transforment [ ] en [X]
\item chercher les éléments non terminés
\end{itemize}
\section{Paragraphe en langue mixte}
\label{sec:org1dff6b6}

Emacs peut aussi servir pour écrire en plusieurs langues.
You can mix French and English in the same buffer.
Ceci te permet de pratiquer les mouvements par mots,
les commandes de recherche, et les remplacements
sélectifs sur certaines expressions seulement.
\section{Fin du fichier de pratique}
\label{sec:org97e2622}

Lorsque tu atteins cette ligne, tu as exploré chaque section de
practice.txt. N'hésite pas à dupliquer des blocs, en insérer de nouveaux,
casser volontairement la mise en forme puis essayer de la réparer en utilisant
les compétences Emacs que tu es en train d'acquérir.
\end{document}
